\chapter{Related Works}

\chapter{Mobile Manipulator}
We conduct or research on a mobile manipulator, lovingly called the \emph{Thing}. It is composed of three main components, on which we elaborate in detail in this chapter. A robot platform is the first, follow by a six degrees of freedom (DOF) robot arm with a gripper at it's end. A force torque sensor is embedded, in the wrist of the gripper. The manipulator is an out of the box product bought from Clearpath, which collaborate with UR and Robotiq and mount the parts on the platform in house.

\section{Ridgeback}
The ridgeback is an omnidirectional robot platform designed by Clearpath for indoor movement and payload carrying tasks, such as autonomous warehousing for example. It is a fully integrated system with sensors, actuation and control and features a native ROS interface. Onboard sensors consist of an IMU and a front facing Hokuyo laser range finder (LIDAR) and a Kinect2 camera and wheel odometry. Optionally, a second, rear facing LIDAR can be mounted for full \unit[360]{\textdegree} coverage.The broad range of sensors, it's flexibility and low drift in odometry makes the ridgeback a suitable and popular platform for research in controlled indoor environments.

\begin{figure}
   \centering
   \includegraphics[width=0.75\textwidth]{images/ridgeback.png}
   \caption{Clearpath Ridgeback}
   \label{pics:ridgeback}
\end{figure}

\begin{table}[h]
\begin{center}
 \caption{Clearpath Ridgeback Specifications}\vspace{1ex}
 \label{tab:ridgeback}
 \begin{tabular}{ll}
 \hline
 Length & \unit[960]{mm}\\
 Width & \unit[793]{mm}\\
 Height & \unit[296]{mm}\\
 Weight & \unit[135]{kg}\\
 Maximum payload & \unit[100]{kg}\\
 Maximum velocity & \unitfrac[1.1]{m}{s}\\
 Average power consumption & \unit[800]{W}\\
 \hline
 \end{tabular}
\end{center}
\end{table}

\section{Universal Robot 10}

\begin{figure}
   \centering
   \includegraphics[width=0.75\textwidth]{images/ur10.png}
   \caption{Universal Robot 10}
   \label{pics:ur10}
\end{figure}

The UR10 is an collaborative industrial robot arm by Universal Robots. It has six DOF and can support payloads up to \unit[10]{kg}.

\begin{table}[h]
\begin{center}
 \caption{Universal Robot 10 Specifications}\vspace{1ex}
 \label{tab:ur10}
 \begin{tabular}{ll}
 \hline
 Reach & \unit[1300]{mm} \\
 Weight & \unit[1.5]{kg}\\
 Repeatability & \unit[0.1]{mm} \\
 Maximum payload & \unit[10]{kg}\\
 Maximum tool velocity & \unitfrac[1]{m}{s}\\
 Degrees of freedom & 6 rotating joints \\
 Average power consumption & \unit[]{W}\\
 \hline
 \end{tabular}
\end{center}
\end{table}

\section{Gripper}

\begin{figure}
   \centering
   \includegraphics[width=0.75\textwidth]{images/robotiq_gripper.jpg}
   \caption{Robotiq 3-Finger Adaptive Robot Gripper}
   \label{pics:robotiq_gripper}
\end{figure}

\begin{table}[h]
\begin{center}
 \caption{Robotiq 3-Finger Adaptive Robot Gripper Specifications}\vspace{1ex}
 \label{tab:robotiq_gripper}
 \begin{tabular}{ll}
 \hline
 Weight & \unit[2.3]{kg}\\
 Repeatability & \unit[0.1]{mm} \\
 Maximum payload (encompassing grip) & \unit[10]{kg}\\
 Gripper opening & \unit[0 to 155]{mm} \\
 Object diameter for encompassing & \unit[20 to 155]{mm}\\
 Grip force & \unit[30 to 70]{N} \\
 Minimum power consumption & \unit[4.1]{W} \\
 Peak power (at maximum gripping force) & \unit[36]{W}\\
 \hline
 \end{tabular}
\end{center}
\end{table}

\section{Force-Torque Sensor}

\begin{figure}
   \centering
   \includegraphics[width=0.75\textwidth]{images/robotiq_ft.jpg}
   \caption{Robotiq FT 300 Force Torque Sensor}
   \label{pics:robotiq_ft}
\end{figure}

\begin{savenotes}
\begin{table}[h]
\begin{center}
 \caption{Robotiq FT 300 Force Torque Sensor Specifications}\vspace{1ex}
 \label{tab:robotiq_ft}
 \begin{tabular}{ll}
 \hline
 \textbf{Measuring range} & \\
 Force $F_x, F_y, F_z$ & \unit[$\pm 300$]{N} \\
 Moment $M_x, M_y, M_z$ & \unit[$\pm 30$]{Nm} \\ \hline
 \textbf{Signal noise}\footnote{Signal noise is the standard deviation of the signal measured over a period of one second.} &\\
 Force $F_x, F_y, F_z$ & \unit[0.1]{N} / \unit[1]{N} \\
 Moment $M_x, M_y$ & \unit[0.05]{Nm} / \unit[0.02]{Nm} \\
 Moment $M_z$ & \unit[0.03]{Nm} / \unit[0.01]{Nm} \\ \hline
 Data output rate & \unit[100]{Hz} \\
 Weight & \unit[300]{g}\\
 \hline
 \end{tabular}
\end{center}
\end{table}
\end{savenotes}

\chapter{Thing Control Structure}

\chapter{Admittance Control}

\chapter{Obstacle Avoidance}

\chapter{Results}

\chapter{Conclusions}

\chapter{Einige wichtige Hinweise zum Arbeiten mit \LaTeX\ }
\label{sec:latexumg}

Nachfolgend wird die Codierung einiger oft verwendeten Elemente
kurz beschrieben. Das Einbinden von Bildern ist in \LaTeX\ nicht
ganz unproblematisch und hängt auch stark vom verwendeten Compiler
ab. Typisches Format für Bilder in \LaTeX\ ist
EPS\footnote{Encapsulated Postscript} oder PDF\footnote{Portable Document Format}.


\section{Gliederungen}
\label{sec:gliederung}

Ein Text kann mit den Befehlen \texttt{\textbackslash
chapter\{.\}}, \texttt{\textbackslash section\{.\}},
\texttt{\textbackslash subsection\{.\}} und \texttt{\textbackslash
subsubsection\{.\}} gegliedert werden.


\section{Referenzen und Verweise}
\label{sec:refverw}

Literaturreferenzen werden mit dem Befehl \texttt{\textbackslash
citep\{.\}} und \texttt{\textbackslash
citet\{.\}} erzeugt. Beispiele: ein Buch \citep{Raibert1986LeggedRobotsThatBalance}, ein Buch und ein Journal Paper \citep{Raibert1986LeggedRobotsThatBalance,Vukobratovic2004ZeroMomentPoint}, ein Konferenz Paper mit Erwähnung des Autors: \citet{Pratt1995SEA}.

Zur Erzeugung von Fussnoten wird der Befehl \texttt{\textbackslash
footnote\{.\}} verwendet. Auch hier ein Beispiel\footnote{Bla
bla.}.

Querverweise im Text werden mit \texttt{\textbackslash label\{.\}}
verankert und mit \texttt{\textbackslash cref\{.\}} erzeugt.
Beispiel einer Referenz auf das zweite Kapitel:
\cref{sec:latexumg}.


\section{Aufzählungen}\label{sec:aufz}

Folgendes Beispiel einer Aufzählung ohne Numerierung,
\begin{itemize}
  \item Punkt 1
  \item Punkt 2
\end{itemize}
wurde erzeugt mit:
\begin{verbatim}
\begin{itemize}
  \item Punkt 1
  \item Punkt 2
\end{itemize}
\end{verbatim}

Folgendes Beispiel einer Aufzählung mit Numerierung,
\begin{enumerate}
  \item Punkt 1
  \item Punkt 2
\end{enumerate}
wurde erzeugt mit:
\begin{verbatim}
\begin{enumerate}
  \item Punkt 1
  \item Punkt 2
\end{enumerate}
\end{verbatim}

Folgendes Beispiel einer Auflistung,
\begin{description}
  \item[P1] Punkt 1
  \item[P2] Punkt 2
\end{description}
wurde erzeugt mit:
\begin{verbatim}
\begin{description}
  \item[P1] Punkt 1
  \item[P2] Punkt 2
\end{description}
\end{verbatim}


\section{Erstellen einer Tabelle}\label{sec:tabellen}

Ein Beispiel einer Tabelle:
\begin{table}[h]
\begin{center}
 \caption{Daten der Fahrzyklen ECE, EUDC, NEFZ.}\vspace{1ex}
 \label{tab:tabnefz}
 \begin{tabular}{ll|ccc}
 \hline
 Kennzahl & Einheit & ECE & EUDC & NEFZ \\ \hline \hline
 Dauer & s & 780 & 400 & 1180 \\
 Distanz & km & 4.052 & 6.955 & 11.007 \\
 Durchschnittsgeschwindigkeit & km/h & 18.7 &  62.6 & 33.6 \\
 Leerlaufanteil & \% & 36 & 10 & 27 \\
 \hline
 \end{tabular}
\end{center}
\end{table}

Die Tabelle wurde erzeugt mit:
\begin{verbatim}
\begin{table}[h]
\begin{center}
 \caption{Daten der Fahrzyklen ECE, EUDC, NEFZ.}\vspace{1ex}
 \label{tab:tabnefz}
 \begin{tabular}{ll|ccc}
 \hline
 Kennzahl & Einheit & ECE & EUDC & NEFZ \\ \hline \hline
 Dauer & s & 780 & 400 & 1180 \\
 Distanz & km & 4.052 & 6.955 & 11.007 \\
 Durchschnittsgeschwindigkeit & km/h & 18.7 &  62.6 & 33.6 \\
 Leerlaufanteil & \% & 36 & 10 & 27 \\
 \hline
 \end{tabular}
\end{center}
\end{table}
\end{verbatim}


\section{Einbinden einer Grafik}\label{sec:epsgraph}

Das Einbinden von Graphiken kann wie folgt bewerkstelligt werden:
\begin{verbatim}
\begin{figure}
   \centering
   \includegraphics[width=0.75\textwidth]{images/k_surf.pdf}
   \caption{Ein Bild.}
   \label{fig:k_surf}
\end{figure}
\end{verbatim}

\begin{figure}
   \centering
   \includegraphics[width=0.75\textwidth]{images/k_surf.pdf}
   \caption{Ein Bild}
   \label{pics:k_surf}
\end{figure}

oder bei zwei Bildern nebeneinander mit:
\begin{verbatim}
\begin{figure}
  \begin{minipage}[t]{0.48\textwidth}
    \includegraphics[width = \textwidth]{images/cycle_we.pdf}
  \end{minipage}
  \hfill
  \begin{minipage}[t]{0.48\textwidth}
    \includegraphics[width = \textwidth]{images/cycle_ml.pdf}
  \end{minipage}
  \caption{Zwei Bilder nebeneinander.}
  \label{pics:cycle}
\end{figure}
\end{verbatim}

\begin{figure}
  \begin{minipage}[t]{0.48\textwidth}
    \includegraphics[width = \textwidth]{images/cycle_we.pdf}
  \end{minipage}
  \hfill
  \begin{minipage}[t]{0.48\textwidth}
    \includegraphics[width = \textwidth]{images/cycle_ml.pdf}
  \end{minipage}
  \caption{Zwei Bilder nebeneinander}
  \label{pics:cycle}
\end{figure}


\section{Mathematische Formeln}\label{sec:math}

Einfache mathematische Formeln werden mit der equation-Umgebung
erzeugt:
\begin{equation}
 p_{me0f}(T_e,\omega_e) \ = \ k_1(T_e) \cdot (k_2+k_3 S^2
 \omega_e^2) \cdot \Pi_{\mathrm{max}} \cdot \sqrt{\frac{k_4}{B}} \, .
 	\label{eq:my_equation}
\end{equation}

Der Code dazu lautet:
\begin{verbatim}
\begin{equation}
 p_{me0f}(T_e,\omega_e) \ = \ k_1(T_e) \cdot (k_2+k_3 S^2
 \omega_e^2) \cdot \Pi_{max} \cdot \sqrt{\frac{k_4}{B}} \, .
\end{equation}
\end{verbatim}

Mathematische Ausdrücke im Text werden mit \$formel\$ erzeugt (z.B.:
$a^2+b^2=c^2$).

Vektoren und Matrizen werden mit den Befehlen \texttt{\textbackslash vec\{.\}} und \texttt{\textbackslash mat\{.\}} erzeugt (z.B. $\vec{v}$, $\mat{M}$).


\section{Weitere nützliche Befehle}\label{sec:div}

Hervorhebungen im Text sehen so aus: \emph{hervorgehoben}. Erzeugt
werden sie mit dem \texttt{\textbackslash epmh\{.\}} Befehl.

Einheiten werden mit den Befehlen \texttt{\textbackslash unit[1]\{m\}} (z.B.~\unit[1]{m}) und \texttt{\textbackslash unitfrac[1]\{m\}\{s\}} (z.B.~\unitfrac[1]{m}{s}) gesetzt.
